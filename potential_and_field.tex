\section{Potential and field}
    \subsection{Gravitational field}
        \paragraph{Gravitational force}
            Force between mass $m$ and $M$ at distance $r$ 
            \begin{align}
                F = \frac{G M m}{r^2}
            \end{align}

            $G$ is gravitational constant.

        \paragraph{Gravitational field strength}
            Defined as 
            \begin{align}
                g = \frac{F}{m}
            \end{align}

        \paragraph{Gravitational potential energy}
            Work done by an affact on a mass to take it to infinity distance wih constant speed.

            Two body system with mass $m$ and $M$.  $m$ 's gravitational energy at $r$ is
            \begin{align}
                U &= \int_{r}^{\infty} (- \frac{G M m}{x^2}) \mathrm{d} x \\
                  &= - G M m (- r^{-1}) |_r^\infty \\
                  &= - \frac{G M m}{r}
            \end{align}

            Potential energy difference between at $r$ and $r + \Delta h$ is $m g \Delta h$
            \begin{align}
                \Delta U &= (- \frac{G M m}{r + \Delta h}) - (- \frac{G M m}{r}) \\
                         &= G M m (\frac{1}{r} - \frac{1}{r + \Delta h}) \\
                         &= m g r^2 (\frac{1}{r} - \frac{1}{r + \Delta h}) \\
                         &= m g \frac{r \Delta h}{r + \Delta h} \\
                         &= mg \Delta h
            \end{align}
        
        \paragraph{Gravitational potential}
            Gravitational potential energy per unit of mass. Defined as 
            \begin{align}
                V = \frac{U}{m} = - \frac{G M}{r}
            \end{align}

    \subsection{Electric field}
        \paragraph{Charge}
            Electron have negative charge, proton have positive charge, neutron have zero charge.

            An electron have charge $- 1.6 \times 10^{-19} \mathrm{C}$. An proton have charge $+ 1.6 \times 10^{-19} \mathrm{C}$.

        \paragraph{Force beween charges}
            Electric force,
            \begin{align}
                F = \frac{k Q q}{r^2}
            \end{align}

            \begin{align}
                k &= \frac{1}{4 \pi \epsilon_0} \\
                  &= \frac{1}{4 \pi \cdot 8.85 \times 10^{-12}} \\
                  &= 8.99 \times 10^9
            \end{align}

        \paragraph{Electric field strength}
            \begin{align}
                E = \frac{F}{q}
            \end{align}

        \paragraph{Electric potential energy}
            Similar to gravitational potential energy.
            \begin{align}
                U = \frac{k Q q}{r}
            \end{align}

        \paragraph{Electric potential}
            Similarly,
            \begin{align}
                V = \frac{U}{q} = \frac{kQ}{r}
            \end{align}
        
        \paragraph{Charged ball shell}
            In a ball shell: charge is 0.

            Outside ball shell:

            Gauss' flux theorem:
            \begin{align}
                \oint E \mathrm{d} S = \frac{\Sigma q}{\epsilon_0}
            \end{align}

            \begin{align}
                & \therefore E \cdot 4 \pi r^2 = \frac{Q}{\epsilon_0} \\
                & \therefore E = \frac{Q}{4 \epsilon_0 \pi r^2} = k \frac{Q}{r^2}
            \end{align}


    \subsection{Equal potential surface}
        Set of points that have equal potential

    \subsection{Field line}
        \paragraph{Direction}
            Go along field line, potential decrease.

    

        